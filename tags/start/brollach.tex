
%%%%%%%%%%%%%%%%%%%%%%%%%%%%%%%%%%%%%%%%%%%%%%%%%%%%%%%%%%%%%%%%%%%%%%%%%%%%%
%%                   Coipcheart                                            %%
%%%%%%%%%%%%%%%%%%%%%%%%%%%%%%%%%%%%%%%%%%%%%%%%%%%%%%%%%%%%%%%%%%%%%%%%%%%%%

\newpage
\chapter*{C\'oipcheart} 
\addcontentsline{toc}{chapter}{C\'oipcheart}

\begin{center}
\textit{This program is free software; \\
you can redistribute it and/or modify it under the terms of the \\
\href{http://www.gnu.org/licenses/gpl.html}{GNU General Public License}
as published by the
\href{http://www.gnu.org/fsf/fsf.html}{Free Software Foundation}; \\
either version 2 of the License, or
(at your option) any later version.}
\vspace{3ex}

\textit{This program is distributed in the hope that it will be useful, \\
but WITHOUT ANY WARRANTY; \\
without even the implied warranty of
MERCHANTABILITY\\ or FITNESS FOR A PARTICULAR PURPOSE. \\
See the GNU General Public License for more details.}
\vspace{5ex}

If n\'{\i}l t\'u in ann an cead\'unas a access\'ail via na nascanna thuas,
scr\'{\i}obh chuig:\\
\vspace{2ex}
Free Software Foundation, Inc.\\
675 Massachusetts Avenue\\
Cambridge, MA 02139, USA
\vspace{7ex}

An Ch\'ead Eagr\'an \copyright \hspace{0.6ex} Bealtaine 2002 \\
\vspace{1ex}
An tEagr\'an seo 
\copyright \hspace{0.6ex} Caoimh\'{\i}n P. \'O Scanaill \the\year \\
\vspace{6ex}
{\LARGE Le f\'ail \'on cl\'ar cinn {\bf \'Ar dTeanga F\'ein}} \\
\vspace{1.5ex}
\href{http://borel.slu.edu/AdTF/index-en.html}{\large\tt http://borel.slu.edu/AdTF/} \\
\vspace{18ex}
{\small Arna chl\'ochur le \LaTeX}
\end{center}


%%%%%%%%%%%%%%%%%%%%%%%%%%%%%%%%%%%%%%%%%%%%%%%%%%%%%%%%%%%%%%%%%%%%%%%%%%%%%
%%                   Giorruchain                                           %%
%%%%%%%%%%%%%%%%%%%%%%%%%%%%%%%%%%%%%%%%%%%%%%%%%%%%%%%%%%%%%%%%%%%%%%%%%%%%%

\newpage
\chapter*{Giorr\'uch\'ain} 
\addcontentsline{toc}{chapter}{Giorr\'uch\'ain}
\begin{tabbing}
\hspace*{35ex}\=\hspace{25ex}\=\kill
\> {\it aid, a1, a2, a3} \> aidiacht \\
\> {\it af} \> ainmfhocal \\
\> {\it b} \> baininscneach \\
\> {\it b1, b2, \ldots} \> ainmfhocal baininscneach \\
\> {\it br} \> briathar \\
\> {\it db} \> dobhriathar \\
\> {\it f} \> firinscneach \\
\> {\it f1, f2, \ldots} \> ainmfhocal firinscneach \\
\> {\it iol} \> iolra \\
\end{tabbing}

%%%%%%%%%%%%%%%%%%%%%%%%%%%%%%%%%%%%%%%%%%%%%%%%%%%%%%%%%%%%%%%%%%%%%%%%%%%%%
%%                   Usaid                                                 %%
%%%%%%%%%%%%%%%%%%%%%%%%%%%%%%%%%%%%%%%%%%%%%%%%%%%%%%%%%%%%%%%%%%%%%%%%%%%%%
\chapter*{{\'U}s\'aid} 
\addcontentsline{toc}{chapter}{\'Us\'aid}

\begin{itemize}
\item There are about 1000 {\em primary entries} given in {\bf boldface}
and followed by lists of related nouns, verbs, adjectives, and
adverbs, respectively.  
\vspace{2ex}
\item The remaining entries are given in plain type and are followed by 
hypertext cross-references (in {\bf boldface}) to one or more primary entries.
\vspace{2ex}
\item Entries are in strict alphabetical order without regard for
spaces or hyphens.
\vspace{2ex}
\item No variant spellings or detailed grammatical information is given.
\vspace{2ex}
\item Bibliographical references are provided for words not
appearing in our standard references: 
{\it Focl\'oir Gaeilge-B\'earla} \cite{OD77}, 
an {\it English-Irish Dictionary} \cite{Ba59},
agus an {\it Collins Gem Irish Dictionary} \cite{Gem95} 
(for modern terminology).
\vspace{2ex}
\item More ideas from thesaurii?
\vspace{2ex}
\item Visit \href{http://borel.slu.edu/AdTF/index-en.html}{{\it \'Ar dTeanga F\'ein}} 
for more details concerning this and other free Irish language software.
\end{itemize}
\endinput
